\chapter{Introduction}
\label{chapter:introduction}

Diabetes is a disease in which the body either does not produce enough insulin or cannot make sufficient use of the insulin that it does produce. Insulin is a critical hormone that allows the glucose from the blood stream to be absorbed into cells. Without it, or a way for the body to properly process it, too much glucose remains in the bloodstream. High amounts of glucose in the bloodstream can lead to several serious complications, including heart disease, vision loss, and kidney disease \cite{CDC:diabetes}. There are two major types of diabetes, \ac{T1D} and \ac{T2D}, both of which require careful management of \ac{BGLs}. %blood glucose levels (BGLs).

In \ac{T1D}, which is the most severe type of diabetes, the pancreas does not produce insulin, which leads to BG levels that are too high \cite{CDC:diabetes}. This is thought to be caused by the destruction of cells in the pancreas by an autoimmune reaction. Of all people diagnosed with diabetes, only about 5-10\% of them have type 1 \cite{CDC:diabetes}. Those with \ac{T1D} typically are diagnosed with it early in life. As of now, there is no known cure or prevention measure for \ac{T1D}.

In \ac{T2D}, the body does not use insulin efficiently enough to keep \ac{BGLs} at normal levels. Type 2 accounts for about 90-95\% of all cases of diabetes \cite{CDC:diabetes}. Fortunately, \ac{T2D} can be prevented or delayed by living a generally healthy lifestyle.

The day-to-day self-management of diabetes is essential for someone with \ac{T1D}. The patient must painstakingly manage their \ac{BGLs} and attempt to prevent them from becoming too high or too low. \ac{T1D} is managed by using external sources of insulin to prevent \ac{BGLs} from becoming too high. However, much care must be put into the amount of insulin that is allowed into the body. Too much insulin can cause \ac{BGLs} to drop too low. When \ac{BGLs} become too high, patients may begin to experience symptoms of hyperglycemia. When \ac{BGLs} drop too low, patients will likely start to feel the effects of hypoglycemia. Both hyperglycemia and hypoglycemia are dangerous and can lead to severe long-term complications if \ac{BGLs} are not promptly corrected.

An important part of diabetes management consists of vigilantly monitoring one's \ac{BGL}. One way this can be done is by using a glucometer, which measures the amount of glucose in a blood sample, typically taken from a fingertip. Another common method of monitoring \ac{BGLs} is by using a \ac{CGM} system, a device capable of measuring blood glucose every few minutes via a subcutaneous sensor \cite{CDC:managing}. A person with \ac{T1D} must also make key decisions about how much insulin to take and when to take it several times per day. They must also make decisions about the quantity and timing of their meals throughout the day.

Diabetes management primarily involves taking actions to correct \ac{BGL} issues as they arise. Once \ac{BGLs} become too high a person will \emph{react} to this by taking insulin. Likewise, a person will \emph{react} to their \ac{BGLs} dropping too low by eating or taking glucose tablets. The process of diabetes management could potentially be simpler and even safer if a more proactive approach was taken instead. Rather than reacting to symptoms of hyperglycemia or hypoglycemia, preemptive steps could be taken to prevent these symptoms altogether.

An important step towards more proactive diabetes management is the ability to forecast future \ac{BGLs} accurately. This information would allow people to make more informed decisions about how to prevent potential hypoglycemic or hyperglycemic events. Efforts to model \ac{BGLs} can be dated back to as early as the 1960s \cite{boutayeb2016}. In recent years, there has been much research into using machine learning algorithms for blood glucose level prediction \cite{plis:maiha14, rubin_falcone:nbeats_bgl, mirshekarian:bgl_pred, bunescu:svr_bgl}. With the widespread adoption of \ac{CGM} systems, and the large amounts of data that they can collect, machine learning has become a more feasible approach to create \ac{BGL} prediction systems. However, even with the ability to forecast \ac{BGLs}, the patient would still need to make a decision about how much to eat to raise their \ac{BGLs} to a normal range or how much to bolus to lower their \ac{BGLs} to a healthy level. This kind of decisions are the primary focus of this thesis.

%However, even with the ability to forecast BGLs, the patient would still need to make a decision about how much to eat/bolus to raise/lower their BGLs to a healthier range or to maintain their current blood glucose level. This is the primary focus of this thesis.

\section{Research Objective}
\label{section:objective}
The objective of the research presented in this thesis is to design and train neural network models capable of providing people with \ac{T1D} recommendations on how many grams of carbohydrates they should eat or how much insulin they should bolus to reach a target \ac{BGL} in the near future. This can be thought of as essentially reversing the \ac{BGL} prediction problem. Instead of predicting \ac{BGLs} in the near future, the objective of this research is to recommend an action that a person should take in order to reach a desired \ac{BGL} in the near future. Previous research in \ac{BGL} prediction \cite{mirshekarian:bgl_pred} has aimed to answer the question of "What will my \ac{BGL} be in an hour if I eat a 30 carbs snack 10 minutes from now?", while this research attempts to answer the question "How many carbs should I eat 10 minutes from now to raise my \ac{BGL} to 140 in an hour?". Two architectures are developed for these tasks, an LSTM-based architecture and an N-BEATS-based architecture. The LSTM-based architecture draws inspiration from the architecture introduced in \cite{mirshekarian:bgl_pred} for the task of \ac{BGL} prediction. The architecture was redesigned to make meal or bolus recommendations using similar input data to that of the original \ac{BGL} prediction system. The N-BEATS-based architecture is a modified version of the original N-BEATS architecture \cite{oreshkin:nbeats}, taking inspiration from another modified version of the N-BEATS architecture that was built for the task of \ac{BGL} prediction \cite{rubin_falcone:nbeats_bgl}. Both architectures are trained on the OhioT1DM dataset \cite{ohiot1dm:marling:kdh18}, which contains data from real people with \ac{T1D}. The goal of this research is to use this real-patient data and proven \ac{BGL} prediction architectures to create accurate, personalized bolus and meal recommendation systems.

%The contribution of this thesis is a general neural network architecture that can be trained to make bolus recommendations and meal recommendations. Two architectures were developed and compared on the tasks of bolus and carbohydrate recommendations, an LSTM-based architecture and an N-BEATS-based architecture. The LSTM-based architecture draws inspiration from the architecture introduced in \cite{mirshekarian:bgl_pred} for the task of BGL prediction. The architecture was redesigned to make meal or bolus recommendations using similar input data to that of the original BGL prediction system. The N-BEATS-based architecture is based on the original N-BEATS architecture \cite{oreshkin:nbeats}, and the modified version that was built for the task of BGL prediction \cite{rubin_falcone:nbeats_bgl}. Our N-BEATS-based architecture makes additions to these architectures which make them better suited for the tasks of bolus and meal recommendations. Both architectures are trained on the OhioT1DM dataset \cite{ohiot1dm:marling:kdh18}, which contains data from real people with T1D. The goal of this research is to use real-patient data and proven BGL prediction architectures to create accurate, personalized bolus and meal recommendation systems.

\section{Thesis Outline}
\label{section:structure}
The rest of this thesis is structured as follows:

\begin{itemize}
	\item Chapter \ref{chapter:background} will compare and contrast this research with other similar research, as well as provide background on some of the core deep learning models that are utilized in this work.
	\item Chapter \ref{chapter:models} will characterize the problems that this research aims to solve as well as the neural architectures that have been designed for each recommendation scenario.
	\item Chapter \ref{chapter:data} will introduce the dataset that is used for this research as well as the pre-processing procedures that have been developed for mitigating the noise in the original raw data.
	\item Chapter \ref{chapter:methods} will explain the procedures for training and evaluating the models, as well as discuss the results of several experimental evaluations.
	\item Chapter \ref{chapter:conclusion} will end this thesis with concluding remarks and suggestions for future research.
\end{itemize}
