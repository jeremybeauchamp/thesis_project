\chapter{Conclusion}
\label{chapter:conclusion}

This thesis has introduced a general \ac{LSTM}-based neural architecture, composed of two chained \ac{LSTM} networks and a fully connected network, with the purpose of training  models for making recommendations with respect to any type of quantitative events that may impact blood glucose levels, in particular, carbohydrate amounts and bolus insulin dosages. A deep residual \ac{N-BEATS}-based architecture was also developed, using the chained \ac{LSTM} networks as a component in its block structure. Experimental evaluations show that the proposed neural architectures substantially outperform a global average baseline as well as a time of day dependent baseline, with the \ac{N-BEATS}-based models outperforming the \ac{LSTM}-based counterparts in all evaluations with inertial examples. The trained models are shown to benefit from transfer learning and from a pre-processing of meal events that anchors their timestamps shortly after their corresponding boluses. Overall, these results suggest that the proposed recommendation approaches hold significant promise for easing the complexity of self-managing blood glucose levels in type 1 diabetes. Potential future research directions include investigating the proposed pre-processing of carbohydrate events for blood glucose level prediction and exploring the utility of the two neural architectures for recommending exercise.